\chapter{Introduction}

% Last updated: 20201025

This small book shows how I won \$10,000 in ten weeks playing
poker. In Computer Programming there's a famous book from
2010 called \textbf{Seven Languages in Seven Weeks}, where seven
programming languages are presented to give you a meaningful taste of
all of them. The languages are Clojure, Haskell, Io, Prolog, Scala,
Erlang and Ruby. I've riffed on that title for this new
poker book, Ten Thousand in Ten Weeks.

Some of this book has appeared in my last three Lessons From Sessions
reports, ``Winter 2019'' (12th Sept 2019), ``Running Hot'' (19th Sept
2019) and ``Early October'' (10th Oct 2019). If you're not a Lessons
From Sessions reader, send me an email and I'll send you those reports
and future Lessons From Sessions. You don't need to read those reports
to appreciate this book, I've duplicated information as necessary and
this book is standalone. My email address is stuart.cooper@gmail.com

% \section*{name} : an unnumbered section that won't appear in the contents
\section*{My Ten Week Spring campaign}

Racehorses don't race every Saturday. They enter a few races, racing
once a fortnight or so, and every so often have a break of 90 days or
more. You can't flog a dead horse, and horses need a break (this is
called a Spell). The horse's group of races are called a preparation
or a \textbf{campaign}, in the first race the horse is ``first-up'',
in the next race ``second-up'' and so on.

As a poker player I don't play every day and I enjoy a week off here
and there; typically when real-world commitments prevent me from
playing my favourite games. I can start or finish a campaign whenever
I like; it's just a start and finish time for a series of session
results. After a breakthrough win at a \$250 Pot Limit Omaha Sit and
Go, I started winning in most of my sessions and felt that I was in
the middle of an excellent campaign.

I'm now finishing my Spring 2019 Campaign with the publication of this
book. My Summer 2019 Campaign starts Friday 25th October, with a trip
to Casino Canberra for their 1/2/5 or 5/5 PLO game.

\section*{Significant Sessions}

% FIXME: check the number of significant sessions in the final copy
% is indeed fourteen

There are fourteen significant sessions in the campaign covered in this
book. A \textbf{significant session} is a poker game where I've
committed two hundred dollars or more. If I play a \$22 pub game with
a \$250 first prize, that game cannot be a significant session; the
money isn't big enough to care about and won't affect my final
result. I didn't play any games of that size in Spring 2019. There
were two sessions I played that did not reach significant status. Both
were 1/1 PLO cash games. In the first game (at Lakes) I was in for
\$150 and cashed out for \$190, winning \$40, in the second game
(at Mortdale) I was in for \$150 and cashed out for \$110, losing
\$40. Those two results cancel each other out and I won't include them
in this book.

% FIXME: but quickly discuss them?

\section*{How to win \$10,000 at poker}

The easiest way of winning \$10,000 at poker is to get lucky in a
single big tournament. Indeed I was at the final table of a 100+
player, \$250 buyin tournament in August, where the first prize was a
guaranteed \$10,000. Because the first prize was guaranteed to be 10k,
that tournament had a very top-heavy payout structure. I finished
fourth for \$750. Along with the cash part of that prize, I won a
ticket to a late-October game with a guaranteed \$75,000 prize
pool. So getting lucky in that tournament will be an even bigger
payout than the \$10,000 this book writes about.

The hard way of winning \$10,000 at poker is to do well at big cash
games. That's exactly how I did it. About 60\% of my \$10,000 came
from Pot Limit Omaha cash games at Casino Canberra; in just three
visits. Another 25\% came from a big Pot Limit Omaha Sit and Go at
Petersham. Just under 10\% came from No Limit Holdem Tournament
Winnings. The important point is if you're going to win steadily at
poker, you'll need to be an excellent cash player. Tournaments just
can't get it done with enough regularity; although if you can get
lucky in a big-field tournament you'll win more money than you will at
a month of cash play. Tournaments are the dream of poker players, cash
games are the reality.

This point is so important that it bears repeating: if you're going to
win big regular money at poker you'll have to do it through Cash.
Tournament final tables get the glamour and the media, big cash games
are where the pros make their actual money.

Tournament results are public information and you can see the results
of the Pros at websites such as The Hendon Mob. Cash results are
typically private and unverifiable. You may know that poker legend
Barry Greenstein donates his Tournament Winnings to charity. He keeps
his cash winnings.

%% \section*{About the author}

% FIXME: Back cover copy?

%% Stuart Cooper wrote the ``Doctor Straight'' column, primarily
%% on Pub Poker Tournaments, in the Australian Poker Weekly newspaper
%% for the entirety of the newspaper's two-and-a-half year run.
%% In the 131 issues of the paper he produced 105 columns.

%% Stuart has three major tournament wins (50\$ buyin, 100+ players)
%% to his credit and many small nightly tournament wins.

%% He lives and plays poker in Sydney.
