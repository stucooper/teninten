\chapter{Every session}

\section*{PSH1: Petersham \$250 PLO Sit and Go}

This is a fortnightly game played on Thursday nights at Petersham.
I played it four times in the Spring Campaign for an agreed first,
a split-first, a split-first and a seventh. The games are sessions
PSH1, PSH2, PSH3 and PSH5.

The buyin is \$250 and there's a dealer provided, so it's \$220 to
the prize pool and \$30 to the house. The game has a solid core of
8 or 9 players, and the game is played under ``spare stack''
conditions. Spare Stack means that if fewer than 10 people are playing
that night, rebuys are allowed until 10 stacks have been sold.

Sit and Gos used to be all the rage in Sydney club poker; going all
the way back to the PokerDome Newtown days. I love Sit and Gos.
Bigger tournaments and Chip Chops (cash games in disguise) have
consigned the Sit and Go to history but here at least is a game that
combines two things I really love: Sit and Gos and Pot Limit
Omaha. Send it.

Some of the players in the game have more money than (poker) sense and
are happy to take early chances and re-buy for another \$250 if they
bust and the rebuy option is still available. I'm happy to take just a
single shot; especially since the game has a generous starting stack
of 100,000 and you don't have to get involved in big confrontations
early on if you don't want to.

This game normally ends up with 11 entries in the prizepool and pays
first place \$1520 and second place \$900 (first gets 63\% second gets
37\% this seems a fair split).

In this game I was about a 85\% chip leader and did a handshake deal
with the second place finisher that I was the winner and instead of a
1520/900 split we'd do 1400/1000 with a bonus \$20 tip to the
dealer. Second place was very happy to take this deal and I was that
night's winner and up \$1,150 for my night's work.

When you win this game, the dealer offers to take a ``brag picture''
of you with the winning cash on the table spread out in front of
you. I have three such brag pictures in my collection; all of which
I've reposted to my Facebook. Nothing wrong with a bit of pride in a
big Sit and Go win.

\section*{MCN1: Marconi \$330 PLO with Rebuys}

Big Pot Limit Omaha tournaments are rare as hen's teeth. There's
probably only 10 PLO tournaments of \$300+ buyins in the Sydney
Pub/Club scene every year and you don't find that many at The Star or
Crown Melbourne either; they typically turn up only in quarterly event
series.

So I normally make an effort to play a bigger PLO game; this game was
on a Monday night where I normally play social cards at Parramatta RSL
club; I play social cards probably 9 weeks out of 10 but I do cancel
occasionally when there's another event on and a Marconi PLO
tournament qualified as another event.

This game is by far the smallest result of my Spring Sessions (+\$10!)
but one of the most important because it was a rebuy tournament and
I'd rebought twice and was invested for \$990!!. The game paid down to
five players and my fourth placing got me \$1000 for the smallest of
profits.

It's better to win a tiny \$10 than to lose an enormous \$990 so this
result is one of the most important of my Spring Sessions. Indeed,
this was the session where I had the most money invested.

The same game is being played next Monday, 28th October. I'm not as
keen to play it this time around (I don't like being stuck a grand in
a tournament) so I'll just see how I feel on the day.

In general, Pot Limit Omaha is better as a cash game not a tournament
game. No Limit Holdem is better as a tournament game not a cash game.
The Marconi Club game was short-stack (30k) slow blinds and played
very slowly. The Petersham structure of a huge stack and faster blinds
works better and was much more to my liking in this campaign.

\section*{CCA1: Casino Canberra 1/2/5 PLO Cash}

In the introduction I said that the reality of winning money at poker
is that you'll be doing most of it by winning cash games and that's
the truth of my Spring Campaign. Welcome to my cash game: The 1/2/5
PLO game at Casino Canberra.

Casino Canberra is a boutique casino. The entire Casino opens between
the hours of midday and 4am; and poker is only run from 7pm to
3:30am. If you feel like casino poker outside of those hours, too bad.

The small Holdem game has 2/3 blinds and a buyin range of \$100 to
\$500. So far so standard. The PLO game is 1/2/5 (it's easy to take
the \$5 straddle when it's only \$5 and everyone else at the table is
doing it) with a minimum buyin of \$200 and \textbf{no maximum buyin}.

The fact there's no maximum buyin means some of the players turn up
with satchels full of chips and just casually add on a few thousand
dollars to their stack when they take a beat and they want to make
sure they still cover the table. The main big stacker at the table is
an Asian player named Lok; he is a true force of big stack Omaha.

The last time I played this game (CCA3) the structure was different;
the game was 5/5 PLO no straddle with a buyin range of \$200 to
\$2,000. I do like having an upper limit on the buyin; the game plays
very, very big and I heard that one night there was a three-way
confrontation and a pot of \$30,000 was awarded to the winner. Wow.

Anyhow in this game my \$200 buyin went ballistic and I cashed out for
just under \$3,000 for a \$2775 win. Send it.

\section*{PSH2: Petersham \$250 PLO Sit and Go}

This game was four weeks after my first win at this game described in
PSH1. I had a family committment on Thursday 29th August so was unable
to play as defending champion from my Thursday 15th August breakout
win.

This was a comfortable game and I had a big chip stack from early on
and never looked like missing the money. The other player in the money
had played well also so I was happy to do a straight split with him;
\$1200 each and \$20 dealer tip. A \$950 win for each of us.
